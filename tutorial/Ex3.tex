\documentclass[]{article}
\usepackage{lmodern}
\usepackage{amssymb,amsmath}
\usepackage{ifxetex,ifluatex}
\usepackage{fixltx2e} % provides \textsubscript
\ifnum 0\ifxetex 1\fi\ifluatex 1\fi=0 % if pdftex
  \usepackage[T1]{fontenc}
  \usepackage[utf8]{inputenc}
\else % if luatex or xelatex
  \ifxetex
    \usepackage{mathspec}
    \usepackage{xltxtra,xunicode}
  \else
    \usepackage{fontspec}
  \fi
  \defaultfontfeatures{Mapping=tex-text,Scale=MatchLowercase}
  \newcommand{\euro}{€}
\fi
% use upquote if available, for straight quotes in verbatim environments
\IfFileExists{upquote.sty}{\usepackage{upquote}}{}
% use microtype if available
\IfFileExists{microtype.sty}{%
\usepackage{microtype}
\UseMicrotypeSet[protrusion]{basicmath} % disable protrusion for tt fonts
}{}
\usepackage[margin=1in]{geometry}
\usepackage{color}
\usepackage{fancyvrb}
\newcommand{\VerbBar}{|}
\newcommand{\VERB}{\Verb[commandchars=\\\{\}]}
\DefineVerbatimEnvironment{Highlighting}{Verbatim}{commandchars=\\\{\}}
% Add ',fontsize=\small' for more characters per line
\usepackage{framed}
\definecolor{shadecolor}{RGB}{248,248,248}
\newenvironment{Shaded}{\begin{snugshade}}{\end{snugshade}}
\newcommand{\KeywordTok}[1]{\textcolor[rgb]{0.13,0.29,0.53}{\textbf{{#1}}}}
\newcommand{\DataTypeTok}[1]{\textcolor[rgb]{0.13,0.29,0.53}{{#1}}}
\newcommand{\DecValTok}[1]{\textcolor[rgb]{0.00,0.00,0.81}{{#1}}}
\newcommand{\BaseNTok}[1]{\textcolor[rgb]{0.00,0.00,0.81}{{#1}}}
\newcommand{\FloatTok}[1]{\textcolor[rgb]{0.00,0.00,0.81}{{#1}}}
\newcommand{\CharTok}[1]{\textcolor[rgb]{0.31,0.60,0.02}{{#1}}}
\newcommand{\StringTok}[1]{\textcolor[rgb]{0.31,0.60,0.02}{{#1}}}
\newcommand{\CommentTok}[1]{\textcolor[rgb]{0.56,0.35,0.01}{\textit{{#1}}}}
\newcommand{\OtherTok}[1]{\textcolor[rgb]{0.56,0.35,0.01}{{#1}}}
\newcommand{\AlertTok}[1]{\textcolor[rgb]{0.94,0.16,0.16}{{#1}}}
\newcommand{\FunctionTok}[1]{\textcolor[rgb]{0.00,0.00,0.00}{{#1}}}
\newcommand{\RegionMarkerTok}[1]{{#1}}
\newcommand{\ErrorTok}[1]{\textbf{{#1}}}
\newcommand{\NormalTok}[1]{{#1}}
\ifxetex
  \usepackage[setpagesize=false, % page size defined by xetex
              unicode=false, % unicode breaks when used with xetex
              xetex]{hyperref}
\else
  \usepackage[unicode=true]{hyperref}
\fi
\hypersetup{breaklinks=true,
            bookmarks=true,
            pdfauthor={Stefan Zins, Matthias Sand and Jan-Philipp Kolb},
            pdftitle={Exercise 3},
            colorlinks=true,
            citecolor=blue,
            urlcolor=blue,
            linkcolor=magenta,
            pdfborder={0 0 0}}
\urlstyle{same}  % don't use monospace font for urls
\setlength{\parindent}{0pt}
\setlength{\parskip}{6pt plus 2pt minus 1pt}
\setlength{\emergencystretch}{3em}  % prevent overfull lines
\setcounter{secnumdepth}{0}

%%% Use protect on footnotes to avoid problems with footnotes in titles
\let\rmarkdownfootnote\footnote%
\def\footnote{\protect\rmarkdownfootnote}

%%% Change title format to be more compact
\usepackage{titling}

% Create subtitle command for use in maketitle
\newcommand{\subtitle}[1]{
  \posttitle{
    \begin{center}\large#1\end{center}
    }
}

\setlength{\droptitle}{-2em}
  \title{Exercise 3}
  \pretitle{\vspace{\droptitle}\centering\huge}
  \posttitle{\par}
  \author{Stefan Zins, Matthias Sand and Jan-Philipp Kolb}
  \preauthor{\centering\large\emph}
  \postauthor{\par}
  \predate{\centering\large\emph}
  \postdate{\par}
  \date{2 Februar 2016}



\begin{document}

\maketitle


\begin{center}\rule{0.5\linewidth}{\linethickness}\end{center}

\section{Exercise 3a}\label{exercise-3a}

\begin{enumerate}
\def\labelenumi{\arabic{enumi}.}
\item
  Download the dataset for
  \href{http://www.europeansocialsurvey.org/data/country.html?c=germany}{Germany}
  of the 5th ESS-Round (SDDF File and Sampling Data)
\item
  Create a \texttt{svydesign} object to estimate the mean of the
  variable \texttt{agea}
\item
  To acknowledge that the sample has been collected by a multi stage
  design, estimate the design effect of your estimate above using the
  PSU-Indicator variable (Use the
  \href{https://github.com/BernStZi/SamplingAndEstimation/blob/short/lecture/part_2.pdf}{model
  based approach} described on slide 20 of today's lecture)

  \textbf{Advice:} the variable \texttt{PSU} has to be a factor\\
\item
  Calculate the effective sample size
\end{enumerate}

\begin{center}\rule{0.5\linewidth}{\linethickness}\end{center}

\subsubsection{Obtaining MSB, MSW and
\(b^{*}\)}\label{obtaining-msb-msw-and-b}

\begin{Shaded}
\begin{Highlighting}[]
\NormalTok{Ger.d <-}\StringTok{ }\KeywordTok{read.spss}\NormalTok{(}\StringTok{"ESS5DE.spss/ESS5DE.sav"}\NormalTok{,}
                   \DataTypeTok{to.data.frame =} \OtherTok{TRUE}\NormalTok{,}
                   \DataTypeTok{use.value.labels =} \OtherTok{TRUE}\NormalTok{)}
\NormalTok{Ger.ctry <-}\StringTok{ }\KeywordTok{read.spss}\NormalTok{(}\StringTok{"ESS5_DE_SDDF.spss/ESS5_DE_SDDF.por"}\NormalTok{,}
                      \DataTypeTok{to.data.frame =} \OtherTok{TRUE}\NormalTok{, }
                      \DataTypeTok{use.value.labels =} \OtherTok{TRUE}\NormalTok{)}

\KeywordTok{colnames}\NormalTok{(Ger.d)[}\DecValTok{5}\NormalTok{] <-}\StringTok{ "IDNO"}
\NormalTok{Ger <-}\StringTok{ }\KeywordTok{merge}\NormalTok{(Ger.d,Ger.ctry,}\DataTypeTok{by=}\StringTok{"IDNO"}\NormalTok{, }\DataTypeTok{all.x =} \OtherTok{TRUE}\NormalTok{)}
\NormalTok{Ger$PSU <-}\StringTok{ }\KeywordTok{as.factor}\NormalTok{(Ger$PSU)}
\NormalTok{n <-}\StringTok{ }\KeywordTok{nrow}\NormalTok{(Ger)}
\NormalTok{L <-}\StringTok{ }\KeywordTok{length}\NormalTok{(}\KeywordTok{unique}\NormalTok{(Ger$PSU))}
\end{Highlighting}
\end{Shaded}

\begin{center}\rule{0.5\linewidth}{\linethickness}\end{center}

\begin{Shaded}
\begin{Highlighting}[]
\NormalTok{## deffc}
\NormalTok{b.star <-}\StringTok{ }\KeywordTok{sum}\NormalTok{(}\KeywordTok{tapply}\NormalTok{(Ger$dweight,Ger$PSU,}
                \NormalTok{function(x)}\KeywordTok{sum}\NormalTok{(x)^}\DecValTok{2}\NormalTok{))/}\KeywordTok{sum}\NormalTok{(Ger$dweight^}\DecValTok{2}\NormalTok{)}
\CommentTok{# Calculate an anova for the regression model Age by PSU }
\CommentTok{# (Could also be any other Variable)}
\NormalTok{lin.mod <-}\StringTok{ }\KeywordTok{lm}\NormalTok{(}\KeywordTok{as.numeric}\NormalTok{(Ger$agea)~Ger$PSU)}
\NormalTok{SS <-}\StringTok{ }\KeywordTok{anova}\NormalTok{(lin.mod) }
\CommentTok{#  MSB and MSW are the means of SSB and SSW}
\NormalTok{MSB <-}\StringTok{ }\NormalTok{SS$}\StringTok{`}\DataTypeTok{Mean Sq}\StringTok{`}\NormalTok{[}\DecValTok{1}\NormalTok{]}
\NormalTok{MSW <-}\StringTok{ }\NormalTok{SS$}\StringTok{`}\DataTypeTok{Mean Sq}\StringTok{`}\NormalTok{[}\DecValTok{2}\NormalTok{]}
\end{Highlighting}
\end{Shaded}

\begin{center}\rule{0.5\linewidth}{\linethickness}\end{center}

\begin{itemize}
\itemsep1pt\parskip0pt\parsep0pt
\item
  Execute the following
  \href{https://raw.githubusercontent.com/BernStZi/SamplingAndEstimation/short/tutorial/Samples_for_EX3b.R}{R-Script:}
  to generate a Multistage- and a Cluster- Sample for the
  belgianmunicipalities data set
\item
  Your workspace now contains the datasets \texttt{income},
  \texttt{Data.be} and \texttt{Data.be2}. \texttt{income} resembles a
  dataset that
\item
  Estimate the mean income from both samples using the \texttt{survey}
  package and compare the results
\end{itemize}

\end{document}
